The \WFD algorithm does not separate old and new frontiers and 
therefore, has to detect them with each execution. Therefore, \WFD's search
domain is bounded to the whole map. In contrast, \FFD algorithm is able to detect new frontiers
only. In order to enable \FFD eliminating previously detected frontiers, it has
to persistently perform a maintenance. 
%The general maintenance algorithm is
%shown in Algorithm \ref{alg:wfd_maintenance}. This algorithm is quite similar
%the \FFD's maintenance algorithm (Algorithm \ref{alg:ffd_maintenance_stage})
% but with a slight modification (active area as input argument). 
The general maintenance algorithm is shown in Chapter \ref{chap:ffd} (Algorithm
\ref{alg:ffd_maintenance_stage}). In this
chapter we discuss the general concepts of the maintenance process
(Section \ref{section:general_maintenenace_concepts}) and present two
algorithms which represent a hybrid version of both \WFD and \FFD (the
former is discussed in Section \ref{section:wfd_incremental} and the
latter is discussed in Section \ref{section:wfd_incremental_parallel}).

\section{General Maintenance Concepts}
\label{section:general_maintenenace_concepts}

In Section \ref{section:ffd_maintaining_previously_detected_frontiers}, we
showed and explained the maintenance stage of the \FFD algorithm (Algorithm
\ref{alg:ffd_maintenance_stage}). This process can be generalized
and applied to other frontier detection algorithms. By applying maintenance, we
can separate between the two parts that any frontier detection needs to handle:
detection of new frontiers and elimination of existing frontiers. 

The presented maintenance algorithm (Algorithm \ref{alg:ffd_maintenance_stage})
is not bounded to a specific frontier detection algorithm. The algorithm gets
its inputs and acts independently. We encourage the reader
to find more details in Section
\ref{section:ffd_maintaining_previously_detected_frontiers}.
By integrating the maintenance algorithm into \WFD, the frontier search domain
can now be reduced (similar to \FFD). 
%\begin{algorithm}[htbp]
\caption{Generic Frontier Maintenance}
\label{alg:wfd_maintenance}
\begin{algorithmic}[1]

\Require $NewFrontiers$
\Comment{list of new detected frontiers}
\Require $frontiersDB$
\Comment{data-structure that contains last known frontiers}
\Require $activeArea$
\Comment{data-structure that contains all points that lie inside the active
area}

\Function{Maintenance}{$activeArea$,$NewFrontiers,frontiersDB$}

\ForAll {Point $p \in ActiveArea$}
\Comment{eliminate previously detected frontiers}
	\If {$p$ is a frontier cell} 

		\State{}		
		\Comment{split the current frontier into two partial frontiers}
		\State{get the frontier $f \in frontiersDB$ which enables $p \in f$}
		
		\State{$f_1 \gets f[1\ldots p]$}
		\State{$f_2 \gets f[(p+1)\ldots |f|]$}
		\State{remove $f$ from $frontiersDB$}
		\State{add $f1$ and $f2$ to $frontiersDB$}
	\EndIf
\EndFor



\ForAll {Frontier $f \in NewFrontiers$}
\Comment{store new detected frontiers}

	\If {$f$ overlaps with an existing frontier $existFrontier$}
		\State{$merged \gets f \cup existFrontier$}
		\State{remove $existFrontier$ from $frontiersDB$}
		\State{add $merged$ to $frontiersDB$}
	\Else 
		\State{create a new index and add $f$ to $frontiersDB$}	
	\EndIf
\EndFor

\EndFunction
\end{algorithmic}
\end{algorithm}


For instance, we integrated the maintenance mechanism into \WFD
algorithm and created two advanced versions of it which are presented in the
following sections. The first, an incremental version of \WFD is called
\WFDINC, and the second, a parallel version which is called \WFDIP.

\section{Incremental Wavefront Frontier Detector}
\label{section:wfd_incremental}
Our first \WFD improvement is called \WFDINC (Algorithm
\ref{alg:WavefrontFrontierDetectorIncremental}). In this version of \WFD,
instead of searching the whole map for frontier points, the search domain
contains only points that lie inside the active area. The idea is based on
the proof in Section \ref{section:ffd_sound_and_complete}; the only region in
the map that contains changes (i.e. contains new frontiers and contains
frontiers that are needed to be eliminated) is the active area. We define the
active area to be the bounding rectangle that is constructed using the farthest
laser readings received from last executions of \WFDINC.

\begin{singlespace}
\begin{algorithm}[htbp]
\caption{Incremental Wavefront Frontier Detector (WFD-INC)}
\label{alg:WavefrontFrontierDetectorIncremental}
\begin{algorithmic}[1]

\Require{$queue_m$} \Comment{queue, used for detecting frontier
points from a given map} 
 
\Require{$pose$} \Comment{current global position of the
robot}

\Require \colorbox{Gray}{$frontiersDB$}
\Comment{data-structure that contains last known frontiers}

%%%%%%%%%%%%%%%%%%%%%%%%%%%%%%%%%%%%%%%%%%%%%%%%%%%%%%%%%%%%%%%
\Function{WFD-INC}{$activeArea,Map,pose, frontiersDB$}

\If {\colorbox{Gray}{best particle index was changed from last execution}}
\label{mark:wfdinc_particle_changed_start}
	\State {\colorbox{Gray}{clear all data from $frontiersDB$}}
	\State {\colorbox{Gray}{$activeArea \gets Map$ }} \Comment{active
	area contains all map points}
\EndIf
\label{mark:wfdinc_particle_changed_end}

\State{$queue_m \gets \emptyset$}
\State{$newFrontiers \gets \emptyset$}

\State {\Call {Enqueue}{$queue_m$, $pose$}}

\State {mark $pose$ as ``Map-Open-List''} 
\Statex{}
\While {$queue_m$ is not empty}

	\State{$p \gets$ \Call {Dequeue}{$queue_m$}}
	 
	\If{$p$ is marked as ``Map-Close-List''}
		\State{continue}
	\EndIf 
	
	\If{$p$ is a frontier point}
		\State{$foundFrontier \gets$ \Call {Extract-Frontier-2D}{$p$}}
		\State{add $foundFrontier$ to $newFrontiers$}
		\State{mark all points of $foundFrontier$ as ``Map-Close-List''}
	\EndIf \newline
	
	\ForAll{$v \in adj(p)$}
		\If{$v$ not marked as \{``Map-Open-List'', ``Map-Close-List''\} 
		 \textbf{and} $v$ has at least one ``Map-Open-Space'' neighbor 
		 \colorbox{Gray}{\textbf{and} $v \in activeArea$} }
		\label{mark:wfdinc_check_if_in_active_area}
		
			\State{\Call {Enqueue}{$queue_m$,$v$}}
		
			\State{mark $v$ as ``Map-Open-List''}
		\EndIf
	\EndFor \newline
	
	\State{mark $p$ as ``Map-Close-List''}
	%\State{\colorbox{Gray}{mark $p$ as ``visited''}}
\EndWhile


\State{\colorbox{Gray}{\Call {Maintain-Frontiers}{$newFrontiers,
frontiersDB,activeArea$}}}
\label{mark:wfdinc_call_maintenance}
\EndFunction
\end{algorithmic}
\end{algorithm}
\end{singlespace}

 
Similar to State-of-the-art frontier detection algorithms, \WFD
has to detect all frontiers in each execution. \WFD has better performance
than State-of-the-art frontier detection algorithm since it has a smaller search
domain. However, comparing to \FFD, it is still limited. If there were
not changes in map orientation, then all previous frontiers should keep their
map coordinates. 

The \WFDINC algorithm exploits this fact and scans only regions
that might contain changes in frontiers (i.e. regions which were hit by the
laser sensors). If a map orientation was changed (Lines
\ref{mark:wfdinc_particle_changed_start}--\ref{mark:wfdinc_particle_changed_end}),
then all previous frontier data is cleared and the
algorithm start detecting all frontiers in the same way as \WFD algorithm.
Another modification from original \WFD algorithm is when checking the adjacent
points of current scanned point (Line
\ref{mark:wfdinc_check_if_in_active_area}). In this version of \WFD, only points
that lie inside the active area are scanned. This reduces the size of the search
domain. The last modification from the original \WFD algorithm is that a
maintenance has to be called, since \WFDINC does not search over the whole map
(Line \ref{mark:wfdinc_call_maintenance}).     

Both \WFD (Algorithm \ref{alg:WavefrontFrontierDetector1}) and \WFDINC
(Algorithm \ref{alg:WavefrontFrontierDetectorIncremental}) share many common code lines.
We highlighted the differences between both algorithms in order to help the reader
to notice the differences between them more easily.
%



\section{Incremental-Parallel Wavefront Frontier Detector}
\label{section:wfd_incremental_parallel}
Our second improvement of \WFD is actually an improvement of \WFDINC and called
\WFDIP. If changes in map orientations happen too often, then \WFDINC's
performance is the same as \WFD (but now including an overhead of maintenance).
Our goal is therefore to artificially reduce the number of changes in map
orientation as much as possible. The solution is simple: \WFDIP keeps a separate
instance of \WFDINC algorithm for each particle. Whenever \WFDIP is executed, it executes
each instance of \WFDINC according to each particle's map data. Therefore, all
instances of \WFDINC never reach lines
\ref{mark:wfdinc_particle_changed_start}--\ref{mark:wfdinc_particle_changed_end}
and keep frontier data between executions is guaranteed. 
\begin{algorithm}[htbp]
\caption{Incremental-Parallel Wavefront Frontier Detector (WFD-IP)}
\label{alg:WavefrontFrontierDetectorIncrementalParallel}
\begin{algorithmic}[1]
\Require{$particles$} \Comment{data-structure that contains all particles of the
SLAM implemantation}
%%%%%%%%%%%%%%%%%%%%%%%%%%%%%%%%%%%%%%%%%%%%%%%%%%%%%%%%%%%%%%%
\ForAll{Particle $p \in particles$}
	\State{\Call{WFD-INC}{$p\to activeArea$, $p\to Map$, $p\to pose$,
	$p\to frontiersDB$}}
\EndFor
\end{algorithmic}
\end{algorithm}

