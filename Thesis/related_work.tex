An outline of the exploration process can be described as follows: while there
is an unknown territory, allocate to each robot a target point to explore and
coordinate team members in order to minimize overlaps. In frontier-based exploration,
targets are drawn from existing \emph{frontiers}, segments that separate known
and unknown regions (see Section \ref{section:definitions} for definitions).

Most literature ignores the computational cost of frontier detection. 
Instead, there are two aspects that are often tackled in existing literature on
exploration: deciding on next target to be explored and coordinating team
members in order to minimize overlaps. The latter is not related to this thesis
and so, we focus on the former, it may indirectly provide
evidence as to the state of the art in frontier detection.

To the best of our knowledge, all of the following works utilize a standard
edge-detection method for computing the frontiers. They therefore recompute
target locations whenever one robot has reached its target location or whenever
a certain distance has been traveled by the robots or after a timeout event.

Yamauchi \cite{yamauchi_frontier-based_1997,yamauchi_frontier-based_1998}
developed the first frontier-based exploration methods. The robots explore an
unknown environment and exchange information with each other when they get new
sensor readings. As a result, the robots build a common map (occupancy grid) in
a distributed fashion. The map is continuously updated until no new regions are
found. In his work, each robot heads to the \emph{centroid}, the center of mass
of the closest \emph{frontier}. All robots navigate to their target
independently while they share a common map. Frontier detection is performed
only when the robot reaches its target.
% Although Yamauchi's method is robust for disconnections, it is clearly not
% optimal because there is no coordination between the team members. Team
% members might cover the same area or even interfere with each other and map
% each other as an obstacle.

Burgard et al. \cite{burgard_collaborative_2000,burgard05tro} focus their
investigation on a probabilistic approach for coordinating a team of robots.
Their method considers the trade-off between the costs of reaching a target and the
utility of reaching that target. Whenever a target point is assigned to a
specific team member, the utility of the unexplored area visible from this
target position is reduced for the other team members. In their work, frontier
detection is carried out only when a new target is to be allocated to a robot.

Wurm et al. \cite{wurm08iros} proposed to coordinate the team members by
dividing the map into segments corresponding to environmental features.
Afterwards, exploration targets are generated within those segments. The result
is that in any given time, each robot explores its own segment. 

In addition, Wurm \cite{kai_2010_email} claims that updating frontiers
frequently is important in a multi-robot setting since the map will be updated
not only by the robot assigned to a given frontier, but also by all the robots
in the team. In reasonably sized environments, Wurm suggests to call
frontier detection on every time-step of the coordination algorithm. In
real world environments, he suggests executing the frontier detection algorithm
according to the traveled distance (i.e. every $0.5m-1m$) or on every second or
whenever a new target is requested. 
%Moreover, he claims that updating frontiers
%frequently is important in a multi-robot team since the map is updated not only
%by the robot assigned to a given frontier but also by all of the robots in the
% team.


Stachniss \cite{stachniss06phd} introduced a method to make use of background
knowledge about typical structures when distributing the team members over the
environment. In his work, Stachniss computes new frontiers when there are new
targets to be allocated. This happens whenever one robot has reached its
designated target location or whenever the distance traveled by the robots or
the elapsed time since last target assignment has exceeded a given threshold.

Berhault et al. \cite{berhault_robot_2003} proposed a combinatorial
auction mechanism where the robots bid on a bunch of targets to navigate. The
robots are able to use different bidding strategies. Each robot has to visit
all the targets that are included in his winning bid. After combining each
robot's sensor readings, the auctioneer omits selected frontier cells as
potential targets for the robots. Frontier detection is performed when creating
and evaluating bids. 

Visser et al. \cite{visser2008wmcnr} investigated how limited
communication range affects multi-robot exploration. They proposed an algorithm
which takes into account wireless constraints when selecting frontier
targets. Visser \cite{visser_2010_email} suggests recomputing frontiers
every 3--4 meters, which in his opinion, has a positive effect.

Juli\'a et al. \cite{julia2012comparison} surveyed the most important
exploration methods according to different aspects such as team
coordination and integration with the SLAM algorithm. None of the
compared exploration strategies takes advantage of real-time frontier detection.

% presented a probabilistic approach for coordinating a team of robots. Their
% method considers the trade-off between the costs of reaching a target and the
% utility of reaching that target. In order to minimize overlaps between team
% members, they defined the utility of a target by the size of the unexplored
% area that can be covered by the robot's sensors upon reaching that target.
% Whenever a target point is assigned to a specific team member, the utility of
% the unexplored area visible from this target position is reduced for the other
% team members. This way a team of multiple robots can minimize overlapping in
% the coverage area.

% Kai et al. \cite{wurm08iros} proposed to take into account the the environmental
% features in order to better distribute the robots in the exploration task and
% hence, reduce overlaps. They divide the map into segments corresponding to
% environmental features (i.e. rooms). Exploration targets are generated within
% the segments (i.e nearest frontier within a specific segement). Frontier
% Detection is performed when one of the robot requests a new target to explore
%\cite{wurm08iros,stachniss06phd,burgard05tro}.


Lau \cite{lau_behavioural_2003} presented a behavioral approach.
The author assumes that all team members start from a known
location. The team members follow the behavior and spread in the
environment while updating a shared map. Frontier detection is called when
the robot plans its next direction of movement.

%Thrun \cite{thrun2003robotic} surveyed different mapping techniques based on
%probablistic approaches. 

% Journal 
Many other works omit details of their frontier detection timing, and thus
provide no evidence for the computational cost of frontier detection. However,
they broadly refer to frontier detection as a central task as part of
exploration, and thus signify its centrality within modern robot exploration
systems.

Sawhney et al. \cite{sawhney_fast_2009} presented an exploration method for both
2D and 3D environments. They showed a  novel visibility per-time metric. Their
method covers nearly the same number of points like other metric methods that
can be found in literature.
However, the time length of the paths is smaller. The outcome is reduced
exploration time. 

Simmons et al. \cite{simmons_coordination_2000} proposed to assign a %
target destination to each robot in a way that that maximizes the expected
map knowledge over time.
They proposed a bid-based heuristic. Each robot estimates
its utility and cost until arriving various targets. According to this
calculation, each robot creates bids. After receiving all bids, a central
agent assigns a target to each robot considering minimization of the overlapping
coverage of the team members.

Bouraqadi et al. \cite{bouraqadi_flocking-based_2009} proposed a flocking-based
approach for solving the exploration problem, where robots act according to the
same set of rules. One of their rules (R5) makes the robot navigate towards the
nearest frontier.

Ko et al. \cite{ko_practical_2003} presented a decision-theoretic approach to
the mapping and exploration problem. Their approach uses an adopted version of
particle filters to estimate the position in the other robot's partial map.

Fox et al. \cite{Fox2010ieee} presented an exploration and mapping
distributed system for multi-robot exploring teams. Their system enables
building a map of the environment and is robust against limited-range communication and
against uncertainty of initial locations of the team members.

Zlot et al. \cite{zlot2002multi} coordinates a team of exploring robots by
applying a market-based approach. The market architecture maximizes
the exploration information gain while minimizing the travel distance in order
to maximize the overall utility.

Kuipers et al. \cite{kuipers1991robot} presented a method for exploration and
mapping for large-scale spatial environments. Their method utilizes the
qualitative properties of large-scale environments before relatively error-prone
geometrical attributes. This in contrary to traditional methods which utilize
sensor data to build a geometrical representation of the environment and than
extract topological structure from the geometric representation. 

Rekleitis et al. \cite{rekleitis1997multi,rekleitis2001multi} presented a
strategy of localization and exploration designed for small-scale and large-scale
environments. Each pair of robots that are able to sense each other work together in order to reduce
localization error. They modeled the free areas as a simple polygon with holes
and use a well-known planar decomposition form. For large-scale areas, they use
trapezoid decomposition and for small-scale areas, they apply a triangulation
of the free space.

Stachniss et al. \cite{stachniss2006speeding} investigated the assignment
of targets to the robot team members. They applied semantic information on the
environment in order to better distribute the explorers over the explored
environment.

Batalin et al. \cite{batalin2003efficient} presented an exploration method which
does not utilize a map for navigating in the environment. Their algorithm uses
markers which are dropped off by the robot and aid the exploration.

Puig \cite{puig2010balanced} presented an algorithm for coordinating the
team members during the exploration task. Their method is based on K-Means (KME) global
optimization strategy. 

Stoeter et al. \cite{stoeter2000robot} presented a mechanism for controlling
a robot team for missions of exploration and surveillance. Controlling the team
members is performed through a hierarchical behavior tree. Their system is
modular and enables adding new kinds of behaviors. 

Ortiz et al. \cite{konolige2003large,konolige2006centibots} presented
\emph{Centibots}, a framework which enables to control a large team of robots performing tasks of
exploration, planning and collaboration in unknown environments.   

Our work on \WFD (Chapter \ref{chap:wfd}) is independent from previous work,
though~\cite{Calisi:2007:MES:1291068.1291076} mentions a frontier detection
algorithm that utilizes breadth-first search, similar to \WFD.
However, \cite{Calisi:2007:MES:1291068.1291076} does not provide details of the
algorithm, nor evaluation of its performance, and so exact similarities and
differences cannot be assessed.
