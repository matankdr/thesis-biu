\subsection*{FFD is Complete and Sound}

\begin{frame}
\label{frame:ffd_complete}
\frametitle{\FFD is Complete}
\only<+> {

\begin{lemma}[\ref{lem:newf}]
\label{lem:newf}
Suppose $f$ is a frontier point at time $t$, which was not a frontier point at
any time $s$, where $s<t$.\\ Then \FFD will mark $f$ as a frontier given
observation $O_t$.
\end{lemma}
}
%\end{frame}

%\begin{frame}
%\frametitle{\FFD is Complete}
\only<+>{
\begin{proof}
\begin{itemize}
\item Towards a contradiction: \FFD did not recognize $f$ as a
frontier
\item $f$ cannot be located wholly within an \textit{unknown} region 
	\begin{itemize}
	  \item because it must have at least one \emph{Open Space} neighbor
	\end{itemize}
\item $f$ cannot be located wholly within a \textit{known} 
    \begin{itemize}
      \item since $f$ is a valid frontier point and hence, its value is $Unknown$
	\end{itemize}
\item Therefore, $f$ must be located on the contour itself.
\item Detection Stage: \FFD scans \emph{all} contour points sequentially
	\begin{itemize}
	  \item and specifically searches for frontier points
	\end{itemize}
\item It follows that $f$ would be detected, \alert{contradicting} the
assumption
\end{itemize}
\end{proof}
}
\end{frame}

\begin{frame}
\frametitle{\FFD is Complete}
\only<+>{
\begin{theorem}[Completeness]
 \label{thm:complete}
Let $f$ be a valid frontier point at time $t$.\\Then \FFD will mark $f$ as a
frontier point given the sequence of observations $\langle O_0,\ldots,O_t\rangle$.
\end{theorem}
}
%\end{frame}

%\begin{frame}
%\frametitle{\FFD is Complete}
\only<+>{
\begin{proof}
\begin{itemize}
\item{\bf Case 1. $f$ is a new frontier point at time $t$.} \\ Trivially,
this case is handled directly by lemma \ref{lem:newf}.
\item {\bf Case 2. $f$ was a new frontier point at time $s$, where $s<t$.} \\
\begin{itemize}
  \item Let $s$ be the earliest time in which $f$ was a frontier.
  \item Based on lemma \ref{lem:newf}, it follows that it was detected at this
  time.
  \item $f$ must be a frontier point that is maintained by \FFD at time $t$.
  \item $f$ can be eliminated only by the maintenance stage.
  \item $f$ is still a valid frontier and hence, was not covered by the
  sensors.
  \item Therefore, $f$ will not be scanned and eliminated in time $t$
  \item $\Rightarrow$ $f$ remains classified as a frontier by \FFD.
\end{itemize}

\item In both cases \FFD will recognize $f$ to be a valid frontier at
time $t$.
\end{itemize}
\end{proof}
}
\end{frame}

\begin{frame}
\label{frame:ffd_sound}
\frametitle{\FFD is Sound}
\only<+>{
\begin{theorem}[Soundness]
\label{thm:sound}
Let $\hat{f}$ be an arbitrary point in the occupancy grid, which is \emph{not}
a frontier at time $t$.\\Then \FFD will not return $\hat{f}$ as a frontier point,
given the sequence of observations $\langle O_0,\ldots,O_t\rangle$.
\end{theorem}
}
%\end{frame}

%\begin{frame}
%\frametitle{\FFD is Sound}
\only<+>{
\begin{proof}
\begin{itemize}
  \item $\hat{f}$ is not \emph{Unknown} or all its neighbors are not \emph{Open
  Space}
  \item {\bf Case 1. $\hat{f}$ is marked as a new frontier.}
  	\begin{itemize}
  	  %\item \FFD detects $\hat{f}$ as a new frontier
  	  \item $\hat{f}$ must be located on the contour \emph{and} detected by
  	  Extraction stage
  	  \item \FFD specificially avoids detecting non-frontier points as
  	  frontiers
  	  \item Case 1 is not possible
  	\end{itemize}
  \item{\bf Case 2. $\hat{f}$ is an old frontier but was not eliminated.}
  	\begin{itemize}
  	  \item $\hat{f}$ is located inside the active area
  	  and was not eliminated.
  	  \item $\hat{f}$ is a point that was covered by the robot's sensors
  	  \item Each point in the map keeps a frontier index (or NULL value)
  	  \item \FFD scans $\hat{f}$ and finds out it has a not NULL frontier index
  	  \item The Maintenance stage will eliminate $\hat{f}$
  	  \item Case 2 is not possible  
  	\end{itemize}
\end{itemize}

\end{proof}
\hyperlink{frame:ffd_sound_complete}{\beamergotobutton{Back to Slide}}
}
\end{frame}
