\abstract{ 
Frontier-based exploration is the most common approach to exploration, a
fundamental problem in robotics. In frontier-based exploration, robots
explore by repeatedly detecting (and moving towards) \emph{frontiers},
the segments which separate the known regions from those unknown. A 
\emph{frontier detection} sub-process examines map and/or sensor readings 
to identify frontiers for exploration. However, most 
frontier detection algorithms process the entire map data. This can be a time
consuming process, which affects the exploration decisions. In this work, we present
several novel frontier detection algorithms that do not process the entire map data,
and explore them in depth. We begin by investigating algorithms that
represent two approaches:  \WFD, a graph search based 
algorithm which examines only known areas, and \FFD, which examines only 
new laser readings data. We analytically examine the complexity of both 
algorithms, and discuss their correctness.  We then improve by 
combining elements of both, to create two additional algorithms,
called \WFDINC and \WFDIP. We empirically evaluate all algorithms,
and show that they are all faster than a state-of-the-art frontier detector 
implementation (by several orders of magnitude).  We additionally contrast
them with each other and demonstrate the \FFD and \WFDIP are faster than
the others by one additional order of magnitude.
}
