\section{Complexity}
\subsection*{WFD}

\begin{frame}
\frametitle{\WFD Complexity}
\begin{itemize}
  \item \WFD is based on Breadth-First Search (BFS) over the map.
  \item \WFD scans all \emph{Open Space} regions for frontier points.
  \item When a frontier point is found, another BFS is executed
  	\begin{itemize}
  	  \item in order to extract the frontier.
  	\end{itemize}
  \item BFS time complexity is linear in size of the search space.
  \item $\Rightarrow$ Linear in size of \emph{area} and \emph{perimeter} of
  \emph{Open Space} regions
\end{itemize}

$$
\order{\underbrace{S\term{open-space}}_{area} +
\underbrace{P\term{open-space}}_{perimeter}}
$$

\end{frame}

\begin{frame}
\frametitle{\WFD Best Case}
\begin{block}{Best Case}
The perimeter of the \openspace regions is minimal relatively to
the area of the \openspace regions
\end{block}
\begin{figure}
  \centering
  \def\bgColor{gray!15}

\def\createBackground{% The graphic
  \draw[fill=\bgColor] (-1.2,-1.2) -- (1.2,-1.2) -- (1.2,1.2) -- (-1.2,1.2) --
  (-1.2,-1.2);
 }


\definecolor{zubi}{RGB}{94,38,18}
\def\emphColor{zubi}
\def\arrowWidth{1pt}	
\begin{tikzpicture}[scale=\MyTikzScale,cap=round]
	  % Local definitions
	  \def\costhirty{0.8660256}
	  \def\cosforthyfive{0.7071067811}
	  % Colors
	  \colorlet{coscolor}{blue}
	
	  % Styles
	  \tikzstyle{axes}=[]
	  \tikzstyle{important line}=[very thick]
	  %\tikzstyle{information text}=[rounded corners,fill=red!10,inner sep=1ex]
	
	  % The graphic
	  %\draw[style=help lines,step=0.5cm] (-1.4,-1.4) grid (1.4,1.4);
	\createBackground 
	\draw[line width=1pt,dashed,fill=white] (0,0) circle (1cm);
	    
	   \draw[style=important line,coscolor]
	    (0,0) -- node[right=2pt,fill=none] {$r$}
	    (0,1);
	\end{tikzpicture}

	%\caption{\WFD Best Case: perimeter of \emph{open-space} regions is small as
	%possible}
	\label{fig:wfd_best_case}
\end{figure}
\end{frame}



\begin{frame}
\frametitle{\WFD Worst Case}
\begin{block}{Worst Case}
\begin{itemize}
  \item Maximize the length of the perimeter 
  	\begin{itemize}
  	  \item while \alert{keeping} the total area of the \openspace regions.
  	\end{itemize}
  \item Use a polygon as an approximation to a circle.
  \item The level of accuracy is determined by $k$, the number
of vertices.
\end{itemize}

\end{block}
\end{frame}



\begin{frame}
\frametitle{\WFD Worst Case}
\begin{figure}
\centering
\only<+>{\def\bgColor{gray!15}

\def\createBackground{% The graphic
  \draw[fill=\bgColor] (-1.2,-1.2) -- (1.2,-1.2) -- (1.2,1.2) -- (-1.2,1.2) --
  (-1.2,-1.2);
 }


\definecolor{zubi}{RGB}{94,38,18}
\def\emphColor{zubi}
\def\arrowWidth{1pt}
\begin{tikzpicture}[scale=\MyTikzScale,cap=round]
  % Local definitions
  \def\costhirty{0.8660256}
  \def\cosforthyfive{0.7071067811}
  % Colors
  \colorlet{coscolor}{blue}

  % Styles
  \tikzstyle{axes}=[]
  \tikzstyle{important line}=[very thick]
  %\tikzstyle{information text}=[rounded corners,fill=red!10,inner sep=1ex]

  \createBackground
    
%     \path (-0.2,-0.2) coordinate (A) (0.2,-0.2) coordinate (B)
%           (0.2,0.2) coordinate (C) (-0.2,0.2) coordinate (D);
%         \draw[fill=white] (A)--(B)--(C)--(D)--cycle;
        
    \newdimen \R
    \R=0.4cm
     \draw[fill=white] (45:\R) \foreach \x in {0,45,135,...,405} {
     	-- (\x:\R)
     } -- cycle (135:\R);

   \draw[style=important line,coscolor]
    (0,0) -- node[right=2pt,fill=none] {$r$}
    (90:\R);
    
    
    
    
    \foreach \x in {45,135,...,405} {
%  		\prev=67.5;
     	\path (\x:\R) coordinate (P1) ;
     	\path (45+\x:\R+0.7cm) coordinate (P2);
     	\path (90+\x:\R) coordinate (P3);
     	\draw[fill=white, dashed, line width=1pt] (P1) -- (P2) -- (P3);
    }
% 	\draw[fill=white] (22.5:\R) \foreach \x in {67.5,112.5,...,382.5} {
%      	-- (\x:\R)
%      }	
    
    %\draw[fill=white, line width=1pt,dashed] (A)--(0,-1.2)--(B);
    %\draw[fill=white] (B)--(BC)--(C)--cycle;
    %\draw[fill=white] (C)--(CD)--(D)--cycle;
    %\draw[fill=white] (D)--(DA)--(A)--cycle;

    \draw[style=important line,\emphColor]
    (-45:\R) -- node[left=0pt,fill=none] {}
    (45:\R);
    
    \draw[\emphColor]
    (\R-0.11cm,0) -- node[left=5pt,above=0pt,fill=none] {}
    (0.7cm+\R,0);
    
    \node[anchor=east] at (\R+\R,0) (src_h) {};
  	\node[color=\emphColor,anchor=west] at (45:1.3cm) (dst_h) {$h_{4}$};
  	\node[anchor=east] at (-30:0.866*\R) (src_b) {};
  	\node[color=\emphColor,anchor=west] at (-45:1.3cm) (dst_b) {$b_{4}$};
  	\draw[\emphColor,line width=\arrowWidth] (dst_h) edge[out=180,in=0,<->]
  	(src_h); 
  	\draw[\emphColor,line width=\arrowWidth] (dst_b) edge[out=180,in=0,<->]
  	(src_b);
  	
  	\node[anchor=east] at (\R*0.5,1.5*\R) (src_l) {};
  	\node[color=\emphColor,anchor=west] at (65.5:1.1cm) (dst_l) {$l_{4}$};
  	\draw[\emphColor,line width=\arrowWidth] (dst_l) edge[out=180,in=0,<->]
  	(src_l); \draw[fill=white, \emphColor,line width=2pt] (P1) -- (P2); %l4
  	
%     \draw[style=important line,red]
%     (0.3cm,0) -- node[left=5pt,above=0pt,fill=none] {$h_4$}
%     (2.6*\R,0);
    
\end{tikzpicture} \caption{$k=4$}}
\only<+>{\def\bgColor{gray!15}

\def\createBackground{% The graphic
  \draw[fill=\bgColor] (-1.2,-1.2) -- (1.2,-1.2) -- (1.2,1.2) -- (-1.2,1.2) --
  (-1.2,-1.2);
 }


\definecolor{zubi}{RGB}{94,38,18}
\def\emphColor{zubi}
\def\arrowWidth{1pt}
\begin{tikzpicture}[scale=\MyTikzScale,cap=round]
  % Local definitions
  \def\costhirty{0.8660256}
  \def\cosforthyfive{0.7071067811}
  % Colors
  \colorlet{coscolor}{blue}

  % Styles
  \tikzstyle{axes}=[]
  \tikzstyle{important line}=[very thick]
  %\tikzstyle{information text}=[rounded corners,fill=red!10,inner sep=1ex]

 	\createBackground   
    
    
    \newdimen \R
    \R=0.4cm
     \draw[fill=white] (22.5:\R) \foreach \x in {67.5,112.5,...,382.5} {
     	-- (\x:\R)
     } -- cycle (112.5:\R);

   \draw[style=important line,coscolor]
    (0,0) -- node[right=2pt,fill=none] {$r$}
    (0,0.37);
    
    
    \foreach \x in {22.5,67.5,112.5,...,381.5} {
%  		\prev=67.5;
     	\path (\x:\R) coordinate (P1);
     	\path (22.5+\x:\R+0.7cm) coordinate (P2);
     	\path (45+\x:\R) coordinate (P3);
     	\draw[fill=white, dashed, line width=1pt] (P1) -- (P2) -- (P3);
     	
     	%\draw (P3) -- (P1);
%      	\path (\x:\R) coordinate (x2\x)
    	%\path (0.5*\x+0.5*(\x+45):\R) coordinate (curr)
    	%\draw[fill=white] (prev) -- (curr)
    	%\draw[fill=white] (\x:\R) -- (\x:\R);
    	%\draw (0.5*(\prev+\x):\R) -- (\x:\R);
    }
    
% 	\draw[fill=white] (22.5:\R) \foreach \x in {67.5,112.5,...,382.5} {
%      	-- (\x:\R)
%      }	
    
    %\draw[fill=white, line width=1pt,dashed] (A)--(0,-1.2)--(B);
    %\draw[fill=white] (B)--(BC)--(C)--cycle;
    %\draw[fill=white] (C)--(CD)--(D)--cycle;
    %\draw[fill=white] (D)--(DA)--(A)--cycle;
\draw[style=important line,\emphColor]
    (-22.5:\R) -- node[left=0pt,fill=none] {}
    (22.5:\R);
    
%     \draw[style=important line,red]
%     (0:\R+0.1) -- node[left=10pt,above=1pt,fill=none] {$h_8$}
%     (\R+0.60cm,0);
    
    \draw[\emphColor]
    (\R-0.03cm,0) -- node[left=5pt,above=-5pt,fill=none] {}
    (0.7cm+\R,0);

	\node[anchor=east] at (\R+\R,0) (src_h) {};
  	\node[color=\emphColor,anchor=west] at (45:1.3cm) (dst_h) {$h_{8}$};
  	\node[anchor=east] at (-10:\R) (src_b) {};
  	\node[color=\emphColor,anchor=west] at (-45:1.3cm) (dst_b) {$b_{8}$};
  	\draw[\emphColor,line width=\arrowWidth] (dst_h) edge[out=180,in=0,<->]
  	(src_h); 
  	\draw[\emphColor,line width=\arrowWidth] (dst_b)
  	edge[out=180,in=0,<->] (src_b);
    
    \node[anchor=east] at (\R*0.35,1.5*\R) (src_l) {};
  	\node[color=\emphColor,anchor=west] at (67.5:1.1cm) (dst_l) {$l_{8}$};
  	\draw[\emphColor,line width=\arrowWidth] (dst_l) edge[out=180,in=0,<->]
  	(src_l); 
  	\draw[fill=white, \emphColor,line width=\arrowWidth] (67.5:\R) -- (90:\R+0.7cm); %l8
    
\end{tikzpicture}
 \caption{$k=8$}}
\only<+>{\def\bgColor{gray!15}

\def\createBackground{% The graphic
  \draw[fill=\bgColor] (-1.2,-1.2) -- (1.2,-1.2) -- (1.2,1.2) -- (-1.2,1.2) --
  (-1.2,-1.2);
 }


\definecolor{zubi}{RGB}{94,38,18}
\def\emphColor{zubi}
\def\arrowWidth{1pt}
\begin{tikzpicture}[scale=\MyTikzScale,cap=round]
  % Local definitions
  \def\costhirty{0.8660256}
  \def\cosforthyfive{0.7071067811}
  % Colors
  \colorlet{coscolor}{blue}

  % Styles
  \tikzstyle{axes}=[]
  \tikzstyle{important line}=[very thick]
  %\tikzstyle{information text}=[rounded corners,fill=red!10,inner sep=1ex]

	\createBackground    
%     \path (-0.2,-0.2) coordinate (A) (0.2,-0.2) coordinate (B)
%           (0.2,0.2) coordinate (C) (-0.2,0.2) coordinate (D);
%         \draw[fill=white] (A)--(B)--(C)--(D)--cycle;
    
    % top triangle
    \path (0,-1.2) coordinate (ABC) (1.2,0) coordinate (BC)
          (0,1.2) coordinate (CD) (-1.2,0) coordinate (DA);
    
    \newdimen \R
    \R=0.4cm
     \draw[fill=white] (11.25:\R) \foreach \x in {33.75,56.25,...,371.25} {
     	-- (\x:\R)
     } -- cycle (101.25:\R);

   \draw[style=important line,coscolor]
    (0,0) -- node[right=2pt,fill=none] {$r$}
    (90:\R);
    
    
    \foreach \x in {33.75,56.25,...,371.25}{%{22.5,67.5,112.5,...,381.5} {
%  		\prev=67.5;
     	\path (\x:\R) coordinate (P1);
     	\path (11.25+\x:\R+0.7cm) coordinate (P2);
     	\path (22.5+\x:\R) coordinate (P3);
     	\draw[fill=white, dashed, line width=1pt] (P1) -- (P2) -- (P3);
     	
     	%\draw (P3) -- (P1);
%      	\path (\x:\R) coordinate (x2\x)
    	%\path (0.5*\x+0.5*(\x+45):\R) coordinate (curr)
    	%\draw[fill=white] (prev) -- (curr)
    	%\draw[fill=white] (\x:\R) -- (\x:\R);
    	%\draw (0.5*(\prev+\x):\R) -- (\x:\R);
    }
    
% 	\draw[fill=white] (22.5:\R) \foreach \x in {67.5,112.5,...,382.5} {
%      	-- (\x:\R)
%      }	
    
    %\draw[fill=white, line width=1pt,dashed] (A)--(0,-1.2)--(B);
    %\draw[fill=white] (B)--(BC)--(C)--cycle;
    %\draw[fill=white] (C)--(CD)--(D)--cycle;
    %\draw[fill=white] (D)--(DA)--(A)--cycle;
\draw[style=important line,\emphColor]
    (-11.25:\R) -- node[left=0pt,fill=none] {}
    (11.25:\R);
    
%     \draw[style=important line,red]
%     (0:\R+0.1) -- node[left=10pt,above=1pt,fill=none] {$h_8$}
%     (\R+0.60cm,0);
    
    \draw[\emphColor]
    (\R-0.01cm,0) -- node[left=5pt,above=-5pt,fill=none] {}
    (0.7cm+\R,0);
    
    \node[anchor=east] at (\R+\R,0) (src_h) {};
  	\node[color=\emphColor,anchor=west] at (45:1.3cm) (dst_h) {$h_{16}$};
  	\node[anchor=east] at (-5:\R) (src_b) {};
  	\node[color=\emphColor,anchor=west] at (-45:1.3cm) (dst_b) {$b_{16}$};
  	\draw[\emphColor,line width=\arrowWidth] (dst_h) edge[out=180,in=0,<->]
  	(src_h); 
  	\draw[\emphColor,line width=\arrowWidth] (dst_b) edge[out=180,in=0,<->]
  	(src_b);
    
    \node[anchor=east] at (\R*0.20,1.5*\R) (src_l) {};
  	\node[color=\emphColor,anchor=west] at (80:1.1cm) (dst_l) {$l_{16}$};
  	\draw[\emphColor,line width=\arrowWidth] (dst_l) edge[out=180,in=0,<->]
  	(src_l); \draw[fill=white, \emphColor,line width=2pt] (78.75:\R) -- (90:\R+0.7cm); %l8
    
\end{tikzpicture}
 \caption{$k=16$}}
\only<+>{\def\bgColor{gray!15}

\def\createBackground{% The graphic
  \draw[fill=\bgColor] (-1.2,-1.2) -- (1.2,-1.2) -- (1.2,1.2) -- (-1.2,1.2) --
  (-1.2,-1.2);
 }


\definecolor{zubi}{RGB}{94,38,18}
\def\emphColor{zubi}
\def\arrowWidth{1pt}
\begin{tikzpicture}[scale=\MyTikzScale,cap=round]
  % Local definitions
  \def\costhirty{0.8660256}
  \def\cosforthyfive{0.7071067811}
  % Colors
  \colorlet{coscolor}{blue}

  % Styles
  \tikzstyle{axes}=[]
  \tikzstyle{important line}=[very thick]
  %\tikzstyle{information text}=[rounded corners,fill=red!10,inner sep=1ex]

  % The graphic
  \createBackground   
    
%     \path (-0.2,-0.2) coordinate (A) (0.2,-0.2) coordinate (B)
%           (0.2,0.2) coordinate (C) (-0.2,0.2) coordinate (D);
%         \draw[fill=white] (A)--(B)--(C)--(D)--cycle;
        
    \newdimen \R
    \R=0.4cm
    
     \draw[fill=white] (5.625:\R) \foreach \x in {16.875,22.56,...,365.62} {
     	-- (\x:\R)
     } -- cycle (95.625:\R);

   \draw[style=important line,coscolor]
    (0,0) -- node[right=2pt,fill=none] {$r$}
    (90:\R);
    
    
    \foreach \x in {5.625,11.25,...,365.62} {
%  		\prev=67.5;
     	\path (\x:\R) coordinate (P1);
     	\path (5.625+\x:\R+0.7cm) coordinate (P2);
     	\path (11.25+\x:\R) coordinate (P3);
     	\draw[fill=white, dashed, line width=1pt] (P1) -- (P2) -- (P3);
     	
     	%\draw (P3) -- (P1);
%      	\path (\x:\R) coordinate (x2\x)
    	%\path (0.5*\x+0.5*(\x+45):\R) coordinate (curr)
    	%\draw[fill=white] (prev) -- (curr)
    	%\draw[fill=white] (\x:\R) -- (\x:\R);
    	%\draw (0.5*(\prev+\x):\R) -- (\x:\R);
    }
    
% 	\draw[fill=white] (22.5:\R) \foreach \x in {67.5,112.5,...,382.5} {
%      	-- (\x:\R)
%      }	
    
    %\draw[fill=white, line width=1pt,dashed] (A)--(0,-1.2)--(B);
    %\draw[fill=white] (B)--(BC)--(C)--cycle;
    %\draw[fill=white] (C)--(CD)--(D)--cycle;
    %\draw[fill=white] (D)--(DA)--(A)--cycle;
\draw[style=important line,\emphColor]
    (-5.625:\R) -- node[left=0pt,fill=none] {$b_{32}$}
    (5.625:\R);
    
%     \draw[style=important line,red]
%     (0:\R+0.1) -- node[left=10pt,above=1pt,fill=none] {$h_8$}
%     (\R+0.60cm,0);
    
    \draw[\emphColor]
    (\R,0) -- node[left=5pt,above=-5pt,fill=none] {}
    (0:0.7cm+\R);
%     (\R-0.01cm,0) -- node[left=5pt,above=-5pt,fill=none] {$h_{32}$}
%     (0.7cm+\R,0);

  	\node[anchor=east] at (\R+\R,0) (src_h){};
  	\node[color=\emphColor,anchor=west] at (45:1.3cm) (dst_h) {$h_{32}$};
  	\node[anchor=east] at (\R,0) (src_b) {};
  	\node[color=\emphColor,anchor=west] at (-45:1.3cm) (dst_b) {$b_{32}$};
  	\draw[\emphColor,line width=\arrowWidth] (dst_h) edge[out=180,in=0,<->]
  	(src_h); \draw[\emphColor, line width=\arrowWidth] (dst_b)
  	edge[out=180,in=0,<->] (src_b);

    \node[anchor=east] at (\R*0.65,1.5*\R) (src_l) {};
  	\node[color=\emphColor,anchor=west] at (60:1.3cm) (dst_l) {$l_{32}$};
  	\draw[\emphColor,line width=\arrowWidth] (dst_l) edge[out=180,in=0,<->]
  	(src_l); 
  	\draw[fill=white, \emphColor,line width=2pt] (67.5:\R) -- (71.125:\R+0.7cm);
  	%l8
    
\end{tikzpicture} \caption{$k=32$}}
\end{figure}
\end{frame}

\begin{frame}
\label{frame:wfd_complexity}
\frametitle{\WFD Worst Case}
Run-time complexity of \WFD in terms of \openspace area: 
$$
  \order{S_{open}\cdot 
  \term{
  1+
  \sqrt{\term{\frac{1}{r\cdot\tan{\frac{\pi}{S_{open}}}}}^2
        -\frac{1}{\tan{\frac{\pi}{S_{open}}}} + r^2}}}
  $$  
\hyperlink{frame:wfd_complexity_details}{\beamergotobutton{Complexity in
Details}}
\end{frame}


\subsection*{FFD}
\begin{frame}
\frametitle{\FFD Complexity}
\begin{itemize}
  \item It may seem that \FFD's complexity is contained within \WFD
  	\begin{itemize}
  	  \item since \FFD searches only inside the active area (\openspace)
  	\end{itemize}
  \item It is not true since \FFD has to persistently run in the background
  \item We analyse the complexity of each stage separatly
\end{itemize}
\end{frame}


\begin{frame}
\label{frame:ffd_complexity}
\frametitle{\FFD Complexity}
\begin{itemize}
  \item $\execTime{c}:=$ the length of the contour in time $t$
  \item $\execTime{n_f}:=$ number of frontiers in the frontier database in time
  $t$
  \item Single execution:
   $$\order{\execTime{c} +
        \log{\execTime{n_f}}}  
   $$
  \item \FFD has to run in the background  
  \begin{itemize}
    \item $l_\omega:=$ frequency of the laser sensor
    \item $t_{m}:=$ worst-case elapsed time between two following map-events
    \item $P_n:=$ number of particles in the SLAM implementation
    $$
    \order{P_n\cdot \term{t_m \cdot l_\omega} \cdot\term{\execTime{c} +
        \log{\execTime{n_f}}}} 
    $$
  \end{itemize}
\end{itemize}
\hyperlink{frame:ffd_complexity_details}{\beamergotobutton{Complexity in
Details}}
\end{frame}