\section{WFD}
% \begin{frame}<beamer>
% \frametitle{Outline}
% \tableofcontents[currentsection,currentsubsection]
% \end{frame}
 \subsection*{Background}

\begin{frame}
\frametitle{\WFD: Wavefront Frontier Detector}
%\begin{itemize}
  \WFD is a \emph{Breadth-First Search} approach for frontier detection:
  %\item \WFD avoids searching unknown regions 
  %\item \WFD scans only known regions
%\end{itemize}
%\end{frame}

%\subsection*{WFD Outline}
%\begin{frame}
%\frametitle{\WFD Outline}
\begin{enumerate}
  \item Enqueue current robot position
  \item Perform Breadth-First Search 
  	\begin{itemize}
  	  \item scan only open-space points that were not previously scanned
  	\end{itemize}
  \item For every dequeued point, check if it is a frontier point
  \begin{itemize}
    \item If True: extract the frontier by using another Breadth-First Search
  \end{itemize}
\end{enumerate}
\end{frame}

\subsection*{Conclusions}
\begin{frame}
\frametitle{\WFD Conclusions}
\begin{itemize}
%  \item Ensures that only known regions are actually scanned 
  \item Frontier points are adjacent to open space points
  	\begin{itemize}
  	  \item All relevant frontiers will be found when \WFD finishes
  	  \item Connectivity of frontier points ensures complete frontier extraction
  	  \item The algorithm does not have to scan the entire grid
  each time 
  	\end{itemize} 
  \item Completeness and Soundness are guranteed by \emph{BFS}
  \item \WFD still searches frontiers in all known space
\end{itemize}
\end{frame}