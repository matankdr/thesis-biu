%%%%%%%%%%%%%%%%%%%%%%%%%%%%%%%%%%%%%%%%%%%%%%%%%%%%%%%%%%%%%%%%%%%%%%%%
%%%%%%%%%%%%%%%%%%%%%% Simple LaTeX CV Template %%%%%%%%%%%%%%%%%%%%%%%%
%%%%%%%%%%%%%%%%%%%%%%%%%%%%%%%%%%%%%%%%%%%%%%%%%%%%%%%%%%%%%%%%%%%%%%%%

%%%%%%%%%%%%%%%%%%%%%%%%%%%%%%%%%%%%%%%%%%%%%%%%%%%%%%%%%%%%%%%%%%%%%%%%
%% NOTE: If you find that it says                                     %%
%%                                                                    %%
%%                           1 of ??                                  %%
%%                                                                    %%
%% at the bottom of your first page, this means that the AUX file     %%
%% was not available when you ran LaTeX on this source. Simply RERUN  %%
%% LaTeX to get the ``??'' replaced with the number of the last page  %%
%% of the document. The AUX file will be generated on the first run   %%
%% of LaTeX and used on the second run to fill in all of the          %%
%% references.                                                        %%
%%%%%%%%%%%%%%%%%%%%%%%%%%%%%%%%%%%%%%%%%%%%%%%%%%%%%%%%%%%%%%%%%%%%%%%%

%%%%%%%%%%%%%%%%%%%%%%%%%%%%%%%%%%%%%%%%%%%%%%%%%%%%%%%%%%%%%%%%%%%%%%%%
%% NOTE: If you are getting compilation errors referring to list      %%
%%       definitions that don't match, you may need to upgrade to a   %%
%%       newer version of the enumitem package. Try going to:         %%
%%                                                                    %%
%%   http://www.ctan.org/tex-archive/macros/latex/contrib/enumitem    %%
%%                                                                    %%
%%       then download the enumitem.sty file from there. Place it in  %%
%%       the same directory as your CV. So long as there are no other %%
%%       conflicts with older packages on your system, hopefully that %%
%%       will fix your compilation problems.                          %%
%%%%%%%%%%%%%%%%%%%%%%%%%%%%%%%%%%%%%%%%%%%%%%%%%%%%%%%%%%%%%%%%%%%%%%%%

%%%%%%%%%%%%%%%%%%%%%%%%%%%% Document Setup %%%%%%%%%%%%%%%%%%%%%%%%%%%%

% Don't like 10pt? Try 11pt or 12pt
\documentclass[10pt]{article}

% The automated optical recognition software used to digitize resume
% information works best with fonts that do not have serifs. This
% command uses a sans serif font throughout. Uncomment both lines (or at
% least the second) to restore a Roman font (i.e., a font with serifs).
\usepackage{times}
\renewcommand{\familydefault}{\sfdefault}

% The OCR software also has a hard time with italics. These commands get
% rid of the two common ways to italicize text in LaTeX. Get rid of them
% to turn italics back on.
%\renewcommand\emph[1]{#1}
%\renewcommand\textit[1]{#1}

% This is a helpful package that puts math inside length specifications
\usepackage{calc}

% Layout: Puts the section titles on left side of page
\reversemarginpar

%
%         PAPER SIZE, PAGE NUMBER, AND DOCUMENT LAYOUT NOTES:
%
% The next \usepackage line changes the layout for CV style section
% headings as marginal notes. It also sets up the paper size as either
% letter or A4. By default, letter was used. If A4 paper is desired,
% comment out the letterpaper lines and uncomment the a4paper lines.
%
% As you can see, the margin widths and section title widths can be
% easily adjusted.
%
% ALSO: Notice that the includefoot option can be commented OUT in order
% to put the PAGE NUMBER *IN* the bottom margin. This will make the
% effective text area larger.
%
% IF YOU WISH TO REMOVE THE ``of LASTPAGE'' next to each page number,
% see the note about the +LP and -LP lines below. Comment out the +LP
% and uncomment the -LP.
%
% IF YOU WISH TO REMOVE PAGE NUMBERS, be sure that the includefoot line
% is uncommented and ALSO uncomment the \pagestyle{empty} a few lines
% below.
%

%% Use these lines for letter-sized paper
\usepackage[paper=letterpaper,
            %includefoot, % Uncomment to put page number above margin
            marginparwidth=1.2in,     % Length of section titles
            marginparsep=.05in,       % Space between titles and text
            margin=1in,               % 1 inch margins
            includemp]{geometry}

%% Use these lines for A4-sized paper
%\usepackage[paper=a4paper,
%            %includefoot, % Uncomment to put page number above margin
%            marginparwidth=30.5mm,    % Length of section titles
%            marginparsep=1.5mm,       % Space between titles and text
%            margin=25mm,              % 25mm margins
%            includemp]{geometry}

%% More layout: Get rid of indenting throughout entire document
\setlength{\parindent}{0in}

\usepackage[shortlabels]{enumitem}

% Simpler bibsections for CV sections
% (thanks to natbib for inspiration)
%
% * For lists of references with hanging indents and no numbers:
%
%   \begin{bibsection}
%       \item ...
%   \end{bibsection}
%
% * For numbered lists of references (with hanging indents):
%
%   \begin{bibenum}
%       \item ...
%   \end{bibenum}
%
%   Note that bibenum numbers continuously throughout. To reset the
%   counter, use
%
%   \restartlist{bibenum}
%
%   at the place where you want the numbering to reset.

\makeatletter
\newlength{\bibhang}
\setlength{\bibhang}{1em}
\newlength{\bibsep}
 {\@listi \global\bibsep\itemsep \global\advance\bibsep by\parsep}
\newlist{bibsection}{itemize}{3}
\setlist[bibsection]{label=,leftmargin=\bibhang}
\newlist{bibenum}{enumerate}{3}
\setlist[bibenum]{resume,label=[\arabic*]}
\setlist*[bibsection,bibenum]{%
        itemindent=-\bibhang,
        itemsep=\bibsep,parsep=\z@,partopsep=0pt,
        topsep=0pt}
\let\oldendbibenum\endbibenum
\def\endbibenum{\oldendbibenum\vspace{-.6\baselineskip}}
\let\oldendbibsection\endbibsection
\def\endbibsection{\oldendbibsection\vspace{-.6\baselineskip}}
\makeatother

%%% Setup header and footer (with page number and possible last page)
%
% The first block sets up pages 2--end
% The second block sets up page 1 formatting
%
%%%
%
% NOTE: comment the +LP lines and uncomment the -LP lines to have page
%       numbers without the ``of ##'' last page reference)
%
% NOTE: uncomment the \pagestyle{empty} line to get rid of all page
%       numbers on pages 2--end. To get rid of page numbers on page 1,
%       comment out the \thispagestyle{plain} line on the first page
%       below.
%       (also make sure includefoot is commented out above)
%
\usepackage{fancyhdr,lastpage}
\pagestyle{fancy}
%\pagestyle{empty}      % Uncomment this to get rid of page numbers
\fancyhf{}\renewcommand{\headrulewidth}{0pt}
\fancyfootoffset{\marginparsep+\marginparwidth}
\newlength{\footpageshift}
\setlength{\footpageshift}
          {0.5\textwidth+0.5\marginparsep+0.5\marginparwidth-2in}

%%%% PAGES 2--9 NUMBERING:
%% These two lines put page number in upper-right corner of pages 2--end
\rhead{Matan Keidar, p.~\arabic{page} of \protect\pageref*{LastPage}}   % +LP
%\rhead{Pavlic, p.~\arabic{page}}                                 % -LP

%% These lines put page number in bottom (center) of pages 2--end
%\lfoot{\hspace{\footpageshift}%
%       \parbox{4in}{\, \hfill %
%                    \arabic{page} of \protect\pageref*{LastPage} % +LP
%%                    \arabic{page}                               % -LP
%                    \hfill \,}}
%%%% END PAGE 2--9 NUMBERING

%%%% PAGE 1 NUMBERING:
\makeatletter
\let\oldps@plain\ps@plain
\renewcommand{\ps@plain}{\oldps@plain%
\renewcommand{\@evenfoot}{\hspace*{-\footpageshift}\hfil %
    p.~\arabic{page} of \protect\pageref*{LastPage} % +LP
%    p.~\arabic{page}                               % -LP
    \hfil}%
\renewcommand{\@oddfoot}{\@evenfoot}}
\makeatother
%%%% END PAGE 1 NUMBERING

% Finally, give us PDF bookmarks and colored links
%
% NOTE: Some OCR software might be negatively affected by hyperlinks. So
%       most employers recommend the draft option here.
%
% (to enable hyperlinks and bookmarks, comment out ``draft'' line;
%  to disable hyperlinks and bookmarks, uncomment ``draft'' line)
\usepackage{color,hyperref}
\definecolor{darkblue}{rgb}{0.0,0.0,0.3}
\hypersetup{colorlinks,breaklinks,
            linkcolor=darkblue,urlcolor=darkblue,
            anchorcolor=darkblue,citecolor=darkblue,
            draft
            }

%%%%%%%%%%%%%%%%%%%%%%%% End Document Setup %%%%%%%%%%%%%%%%%%%%%%%%%%%%


%%%%%%%%%%%%%%%%%%%%%%%%%%% Helper Commands %%%%%%%%%%%%%%%%%%%%%%%%%%%%

%%% HEADING AT TOP OF CURRICULUM VITAE

% The title (name) with a horizontal rule under it
% (optional argument typesets an object right-justified across from name
%  as well)
%
% Usage: \makeheading{name}
%        OR
%        \makeheading[right_object]{name}
%
% Place at top of document. It should be the first thing.
% If ``right_object'' is provided in the square-braced optional
% argument, it will be right justified on the same line as ``name'' at
% the top of the CV. For example:
%
%       \makeheading[\emph{Curriculum vitae}]{Your Name}
%
% will put an emphasized ``Curriculum vitae'' at the top of the document
% as a title. Likewise, a picture could be included:
%
%   \makeheading[\includegraphics[height=1.5in]{my_picutre}]{Your Name}
%
% the picture will be flush right across from the name.
\newcommand{\makeheading}[2][]%
        {\hspace*{-\marginparsep minus \marginparwidth}%
         \begin{minipage}[t]{\textwidth+\marginparwidth+\marginparsep}%
             {\large \bfseries #2 \hfill #1}\\[-0.15\baselineskip]%
                 \rule{\columnwidth}{1pt}%
         \end{minipage}}

%%% SECTION HEADINGS

% The section headings. Flush left in small caps down pseudo-margin.
%
% Usage: \section{section name}
\renewcommand{\section}[1]{\pagebreak[3]%
    \hyphenpenalty=10000%
    \vspace{1.3\baselineskip}%
    \phantomsection\addcontentsline{toc}{section}{#1}%
    \noindent\llap{\scshape\smash{\parbox[t]{\marginparwidth}{\raggedright #1}}}%
    \vspace{-\baselineskip}\par}

%%% LISTS

% This macro alters a list by removing some of the space that follows the list
% (is used by lists below)
\newcommand*\fixendlist[1]{%
    \expandafter\let\csname preFixEndListend#1\expandafter\endcsname\csname end#1\endcsname
    \expandafter\def\csname end#1\endcsname{\csname preFixEndListend#1\endcsname\vspace{-0.6\baselineskip}}}

% These macros help ensure that items in outer-type lists do not get
% separated from the next line by a page break
% (they are used by lists below)
\let\originalItem\item
\newcommand*\fixouterlist[1]{%
    \expandafter\let\csname preFixOuterList#1\expandafter\endcsname\csname #1\endcsname
    \expandafter\def\csname #1\endcsname{\csname preFixOuterList#1\endcsname\let\oldItem\item\def\item{\pagebreak[2]\oldItem}}
    \expandafter\let\csname preFixOuterListend#1\expandafter\endcsname\csname end#1\endcsname
    \expandafter\def\csname end#1\endcsname{\let\item\oldItem\csname preFixOuterListend#1\endcsname}}
\newcommand*\fixinnerlist[1]{%
    \expandafter\let\csname preFixInnerList#1\expandafter\endcsname\csname #1\endcsname
    \expandafter\def\csname #1\endcsname{\let\oldItem\item\let\item\originalItem\csname preFixInnerList#1\endcsname}
    \expandafter\let\csname preFixInnerListend#1\expandafter\endcsname\csname end#1\endcsname
    \expandafter\def\csname end#1\endcsname{\csname preFixInnerListend#1\endcsname\let\item\oldItem}}

% An itemize-style list with lots of space between items
%
% Usage:
%   \begin{outerlist}
%       \item ...    % (or \item[] for no bullet)
%   \end{outerlist}
\newlist{outerlist}{itemize}{3}
    \setlist[outerlist]{label=\enskip\textbullet,leftmargin=*}
    \fixendlist{outerlist}
    \fixouterlist{outerlist}

% An environment IDENTICAL to outerlist that has better pre-list spacing
% when used as the first thing in a \section
%
% Usage:
%   \begin{lonelist}
%       \item ...    % (or \item[] for no bullet)
%   \end{lonelist}
\newlist{lonelist}{itemize}{3}
    \setlist[lonelist]{label=\enskip\textbullet,leftmargin=*,partopsep=0pt,topsep=0pt}
    \fixendlist{lonelist}
    \fixouterlist{lonelist}

% An itemize-style list with little space between items
%
% Usage:
%   \begin{innerlist}
%       \item ...    % (or \item[] for no bullet)
%   \end{innerlist}
\newlist{innerlist}{itemize}{3}
    \setlist[innerlist]{label=\enskip\textbullet,leftmargin=*,parsep=0pt,itemsep=0pt,topsep=0pt,partopsep=0pt}
    \fixinnerlist{innerlist}

% An environment IDENTICAL to innerlist that has better pre-list spacing
% when used as the first thing in a \section
%
% Usage:
%   \begin{loneinnerlist}
%       \item ...    % (or \item[] for no bullet)
%   \end{loneinnerlist}
\newlist{loneinnerlist}{itemize}{3}
    \setlist[loneinnerlist]{label=\enskip\textbullet,leftmargin=*,parsep=0pt,itemsep=0pt,topsep=0pt,partopsep=0pt}
    \fixendlist{loneinnerlist}
    \fixinnerlist{loneinnerlist}

%%% EXTRA SPACE

% To add some paragraph space between lines.
% This also tells LaTeX to preferably break a page on one of these gaps
% if there is a needed pagebreak nearby.
\newcommand{\blankline}{\quad\pagebreak[3]}
\newcommand{\halfblankline}{\quad\vspace{-0.5\baselineskip}\pagebreak[3]}

%%% FORMATTING MACROS

% Uses hyperref to link DOI
\newcommand\doilink[1]{\href{http://dx.doi.org/#1}{#1}}
\newcommand\doi[1]{doi:\doilink{#1}}

% For \url{SOME_URL}, links SOME_URL to the url SOME_URL
\providecommand*\url[1]{\href{#1}{#1}}
% Same as above, but pretty-prints SOME_URL in teletype fixed-width font
\renewcommand*\url[1]{\href{#1}{\texttt{#1}}}

% For \email{ADDRESS}, links ADDRESS to the url mailto:ADDRESS
\providecommand*\email[1]{\href{mailto:#1}{#1}}
% Same as above, but pretty-prints ADDRESS in teletype fixed-width font
%\renewcommand*\email[1]{\href{mailto:#1}{\texttt{#1}}}

%\providecommand\BibTeX{{\rm B\kern-.05em{\sc i\kern-.025em b}\kern-.08em
%    T\kern-.1667em\lower.7ex\hbox{E}\kern-.125emX}}
%\providecommand\BibTeX{{\rm B\kern-.05em{\sc i\kern-.025em b}\kern-.08em
%    \TeX}}
\providecommand\BibTeX{{B\kern-.05em{\sc i\kern-.025em b}\kern-.08em
    \TeX}}
\providecommand\Matlab{\textsc{Matlab}}

%%%%%%%%%%%%%%%%%%%%%%%% End Helper Commands %%%%%%%%%%%%%%%%%%%%%%%%%%%

%%%%%%%%%%%%%%%%%%%%%%%%% Begin CV Document %%%%%%%%%%%%%%%%%%%%%%%%%%%%

\begin{document}
\thispagestyle{plain}
\makeheading[\emph{Curriculum vitae}]{Matan Keidar}

\section{Contact Information}

% NOTE: Mind where the & separators and \\ breaks are in the following
%       table. Table is one row made up of three parboxes. The left
%       parbox has address info, the middle parbox has a vertical bar,
%       and the right parbox has phone and electronic contact
%       information.
%
% MACROS: \rcollength is the width of the right column of the table
%             (adjust it to your liking; default is 1.85in).
%         \spacewidth is width of area between left and right boxes.
%         \spacechar is character used to produce perforated vertical
%             boundary between boxes.
%
\newlength{\rcollength}\setlength{\rcollength}{1.85in}%
\newlength{\spacewidth}\setlength{\spacewidth}{20pt}
\newcommand\spacechar{$|$}
%
\begin{tabular}[t]{@{}p{\textwidth-\rcollength-\spacewidth}@{}p{\spacewidth}@{}p{\rcollength}}%

% Address box
\parbox{\textwidth-\rcollength-\spacewidth}{%
\begin{tabular}{l l}
%\textit{ID:} & 066500992 \\
\textit{Birth:} & June 22\textsuperscript{nd}, 1984 \\
\textit{Address:} &30/B Geula st., Kfar-Saba, 44257 \\
\textit{Mobile:}& 054-3233268 \\
\textit{E-mail:}& \email{matankdr@gmail.com}\\
\textit{WWW:}
&\href{www.cs.biu.ac.il/~keidarm1}{{www.cs.biu.ac.il/$_{\widetilde{~}}$
keidarm1}}
\end{tabular}
}

% Cheesy perforated vertical bar between boxes
% Shorten by removing \spacechar's
%& \parbox{\spacewidth}{\centering
%\spacechar\\\spacechar\\\spacechar\\\spacechar\\\spacechar} &

% Non-snail-mail contact information
%\parbox{\textwidth-\rcollength-\spacewidth}{%
%%\parbox{\rcollength}{
%\textit{Mobile:} 054-3233268 \\
%\textit{Home:} 09-7416631 \\
%\textit{E-mail:} \email{matankdr@gmail.com}\\
%\textit{WWW:} \href{www.cs.biu.ac.il/~keidarm1}{www.cs.biu.ac.il/~keidarm1}}

\end{tabular}

%%
%% In modern CV's, it seems like ``Objective'' is frowned upon. Instead,
%% incorporate it into a well-constructed cover letter. The ``More
%% information'' can go at the end of the CV, but it should not distract
%% from the section giving references available to contact.
%%
%
% \section{Objective}
%
% Full-time position that allows for advanced research in electrical and
% computer engineering (communications, control, software, electronics,
% and sustainability), with a particular focus on complex distributed
% systems (i.e., modeling, analysis, design, and verification)
% \begin{innerlist}
%     \item For more information, see \url{http://www.tedpavlic.com/engjobsearch/}
% \end{innerlist}



% \section{Citizenship}
%
% USA

\section{Education}

\href{http://www.biu.ac.il/}{\textbf{Bar-Ilan University}},
Ramat-Gan
\begin{outerlist}

\item[] M.Sc, 
        \href{http://www.cs.biu.ac.il/}
             {Computer Science}
             \hfill \textbf{2010~--2012}
        \begin{innerlist}
        \item Thesis Topic: \emph{Fast Frontier Detection for Robot Exploration}
        \item Advisor:
              \href{http://www.cs.biu.ac.il/~galk/}
                   {Professor Gal A.~Kaminka}
        \item Area of Study: Robotics, Artificial Intelligence, Algorithms
        \end{innerlist}

\item[] B.Sc.,
        \href{http://www.cs.biu.ac.il/}
             {Computer Science} \hfill \textbf{2006~--2009}
        \begin{innerlist}
        \item Computer Science Extensive Track
        \end{innerlist}
\end{outerlist}

\section{Awards}
\begin{innerlist}
	\item CS Department Scholarship for researchers \hfill \textbf{2010, 2011}
	\item CS Department Excellence Scholarship for undergraduates \hfill
	\textbf{2009}
	\item Dean's Honors \hfill \textbf{2007}
\end{innerlist}

\section{Professional Experience}
\href{http://www.tzi.de/spl/}
     \textit{Bar-Ilan University: RoboCup Project, Team Leader} \hfill
     \textbf{2010}
\begin{outerlist}
\item[]
	\begin{innerlist}
	\item Lead a team of 13 undergraduate and graduate students in development
	of a team of soccer-playing humanoid robots, each using embedded
	linux-computer, cameras and sensors. 
	\item Overall lines of code in the project: 318,091
	\item Team leader of BURST (Bar-Ilan University Robot
	Soccer Team). The team was founded in 2009 and is the first team from Israel
	to compete in RoboCup senior league. 
	\item The team consists of both undergraduate
	students and researches. 
	\end{innerlist}
\end{outerlist}

\halfblankline

\textit{Freelance Programmer} \hfill
\textbf{2006}
\begin{outerlist}
\item[]
	\begin{innerlist}
	\item Designed and developed a system to a commercial company. 
	\item The product performs queries and displays the results in a GUI
        which was designed according to customer needs. 
	\end{innerlist}
\end{outerlist}

\halfblankline


\textit{IDF: Programmer} \hfill \textbf{2003~--2005, 2006}
\begin{outerlist}
\item[]
    \begin{innerlist}
        \item Human-resources systems.
        \item Worked in environments of main-frame (PL/I) and C\#.
    \end{innerlist}
\end{outerlist}

% \section{IDF Military Service}
% \textit{Programmer} \hfill \textbf{2003~--2005, 2006}
% \begin{outerlist}
% \item[] 
%     \begin{innerlist}
%         \item Human-resources systems.
%         \item Worked in environments of main-frame (PL/I) and C\#.
%     \end{innerlist}
% \end{outerlist}


\section{Publications}

\begin{bibenum}
    \item Matan Keidar and Gal A. Kaminka. Robot
    Exploration with Fast Frontier Detection: Theory and Experiments. In
    \emph{Proceedings of the 11th International Joint Conference on Autonomous Agents and Multi-Agent Systems (AAMAS-12)}, 2012.

    \item Matan Keidar, Eran Sadeh-Or, and Gal A. Kaminka. Fast Frontier
    Detection for Robot Exploration. In \emph{The Autonomous Robots
    and Multirobot Systems (ARMS) workshop at AAMAS-11}, 2011.
\end{bibenum}

% Add a little space to nudge next ``Conference Publications'' marginpar
% down to make room for tall ``Submitted Journal Publications''
% marginpar. If there are enough submitted journal publications, this
% space will not be needed (and should be removed).
\vspace{0.1in}


\section{Teaching Experience}

\href{http://www.cs.biu.ac.il}{\textbf{Department of Computer Science, Bar-Ilan
University}},
\begin{outerlist}

\item[] \textit{Teaching Assistant}%
    \hfill \textbf{March 2010~-- Present}
%     \\(sample graded material and student evaluations available upon
%     request)
    \begin{innerlist}
        \item Instructor for \emph{Numerical Methods}
        \begin{innerlist}
        	\item An undergraduate course teaching the basics of numerical
        	computations: error analysis, linear and non-linear equations, 
        	interpolation, approximation, numerical integration, \Matlab.
            %\item Spring~2010, Spring~2011 and Spring~2012.

            % \item Sample student evaluations available upon request.
            %\item Responsible for 4 groups of 2-hours recitation.
            %\item Created assignments and taught recitation hours.
            \item Course material can be found at\\
                \url{https://sites.google.com/site/matankeidarhomepage/89-276}.
        \end{innerlist}

        \halfblankline
		\item Instructor for \emph{Computer Organization (A Programmer's
		Perspective)}
        \begin{innerlist}
            % \item Sample student evaluations available upon request.
            \item An undergraduate course teaching the basics of computer
            organization and structure: CPU design, assembly language, memory
            hierarchy, code \\ optimization, data representation.
            %\item Automn~2011,
            %\item Responsible for 2 groups of 2-hours recitation.
            %\item Created assignments and taught recitation hours.
            \item Course material can be found at\\
                \url{https://sites.google.com/site/matankeidarhomepage/89-230}.
        \end{innerlist}
        \halfblankline
        \item Instructor for \emph{Robotics Workshop}
        \begin{innerlist}
            \item In charge of undergraduate students of a robotic course.
			\item A programming project which its contents change yearly.  
        \end{innerlist}
        \halfblankline
    \end{innerlist}
\end{outerlist}






% \href{http://www.tzi.de/spl/}
%      {Robotics Workshop} \hfill \textbf{2010, 2011}
% \begin{innerlist}
% \item In charge of undergraduate students of a robotic course.
% \item A programming project which its contents change yearly.  





\href{http://engineering.biu.ac.il/}{\textbf{Faculty of Engineering, Bar-Ilan
University}},
\begin{outerlist}

\item[] \textit{Teaching Assistant}%
    \hfill \textbf{October 2011~-- March 2012}
%    \\(sample graded material and student evaluations available upon
%    request)
    \begin{innerlist}
        \item Instructor for \emph{Object-Oriented Programming}
        \begin{innerlist}
        	\item An undergraduate course teaching Object-Oriented programming and
        	concepts, heavy focus on C++. 
            %\item Automn~2011.
            % \item Sample student evaluations available upon request.
            %\item Responsible for 2 groups of 2-hours recitation.
            %\item Created assignments, examinations and taught recitation
            % hours.
            \item Course material can be found at\\
                \url{https://sites.google.com/site/matankeidarhomepage/83-223}.
        \end{innerlist}

        \halfblankline
    \end{innerlist}
\end{outerlist}

% \textbf{Other},
% \begin{outerlist}
% 
% \item[] \textit{Freelance Programmer}%
%     \hfill \textbf{2006}
% %    \\(sample graded material and student evaluations available upon
% %    request)
%     \begin{innerlist}
%         \item Designed and developed a system to a commercial company 
%         \item The product performs queries and displays the results in a GUI
%         which was designed according to customer needs.
%     \end{innerlist}
% \end{outerlist}


\section{Languages}
    \begin{innerlist}
    	%\begin{tabular}{l l}
    	\item Hebrew: Fluent
        \item English: \hspace{1.5pt}Fluent
        \item French: \hspace{0.5pt} Fair
     %\end{tabular}
    \end{innerlist}


% \section{Voluntary Projects}
% 
% \href{http://www.tzi.de/spl/}
%      {RoboCup Project, Team Leader} \hfill \textbf{2010}
% \begin{innerlist}
% \item Team leader of BURST (Bar-Ilan University Robot
% Soccer Team). The team was founded in 2009 and is the first team from Israel
% to compete in RoboCup senior league. 
% \item The team consists of both undergraduate
% students and researches. 
% \end{innerlist}
% 
% \halfblankline
% 
% \href{http://www.tzi.de/spl/}
%      {Robotics Workshop} \hfill \textbf{2010, 2011}
% \begin{innerlist}
% \item In charge of undergraduate students of a robotic course.
% \item A programming project which its contents change yearly.  
%  
% \end{innerlist}





\section{Programming Skills}
Programming Languages:
\begin{outerlist}
\item[] 
	\begin{innerlist} 
		\item C, C$+$$+$, Java, Python, C\#, UNIX shell scripting, SQL, \Matlab
	\end{innerlist}
%	\halfblankline
\end{outerlist}
\halfblankline

Operating Systems:
\begin{outerlist}
\item[] 
	\begin{innerlist}
	    \item Mainly on Linux and other UNIX variants
	    \item Microsoft Windows family 
	\end{innerlist}
\end{outerlist}

\end{document}

%%%%%%%%%%%%%%%%%%%%%%%%%% End CV Document %%%%%%%%%%%%%%%%%%%%%%%%%%%%%

%----------------------------------------------------------------------%
% The following is copyright and licensing information for
% redistribution of this LaTeX source code; it also includes a liability
% statement. If this source code is not being redistributed to others,
% it may be omitted. It has no effect on the function of the above code.
%----------------------------------------------------------------------%
% Copyright (c) 2007, 2008, 2009, 2010, 2011 by Theodore P. Pavlic
%
% Unless otherwise expressly stated, this work is licensed under the
% Creative Commons Attribution-Noncommercial 3.0 United States License. To
% view a copy of this license, visit
% http://creativecommons.org/licenses/by-nc/3.0/us/ or send a letter to
% Creative Commons, 171 Second Street, Suite 300, San Francisco,
% California, 94105, USA.
%
% THE SOFTWARE IS PROVIDED "AS IS", WITHOUT WARRANTY OF ANY KIND, EXPRESS
% OR IMPLIED, INCLUDING BUT NOT LIMITED TO THE WARRANTIES OF
% MERCHANTABILITY, FITNESS FOR A PARTICULAR PURPOSE AND NONINFRINGEMENT.
% IN NO EVENT SHALL THE AUTHORS OR COPYRIGHT HOLDERS BE LIABLE FOR ANY
% CLAIM, DAMAGES OR OTHER LIABILITY, WHETHER IN AN ACTION OF CONTRACT,
% TORT OR OTHERWISE, ARISING FROM, OUT OF OR IN CONNECTION WITH THE
% SOFTWARE OR THE USE OR OTHER DEALINGS IN THE SOFTWARE.
%----------------------------------------------------------------------%