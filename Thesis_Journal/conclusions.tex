% This section contains a summary of the work presented in this paper
% (Section \ref{section:conclusions}) and our plans for future work (Section
% \ref{section:future_work}).

% \subsection{Conclusions}
% \label{section:conclusions}
% The main motivation for our work was practical issues. It all started
% from a joint project with MAFAT.
% The goal of the project was to create an autonomous robot that explores the
% entire floor of our lab.
% However, we found out that the frontier detection module is very time-consuming.  
% It was then that we realized that frontier detection deserves more study.

Frontier-based exploration is the most common approach to solve the exploration
problem. State-of-the-art frontier detection methods process the entire map data
which hangs the exploration system for a few seconds with every call to the
frontier detection algorithm. 

In this work we present four novel faster frontier detectors, \WFD, \FFD,
\WFDINC and \WFDIP. The first algorithm, a graph-based search, processes the map
points which have already been scanned by the robot sensors and therefore, does
not process unknown regions in each run (though it grows slower as more area is
known). The second algorithm, a laser-based approach for frontier detection,
only processes new laser readings which are received in real time, eliminating
also much of the known search area. However, maintaining previous frontiers
knowledge requires tight integration with the mapping component, which may not
be straightforward. The third and fourth algorithms are a combined approach of
both \WFD and \FFD. Both algorithms search for frontiers within the known
regions but their search space is still smaller than \WFD's search space since
they search for frontiers only in the regions that were covered by the robot
sensors since their last execution. 

We describe efficient implementation for all algorithms and compare them
empirically. \FFD and \WFDIP are shown to outperform \WFD, \WFDINC by 1--2
orders of magnitude. In addition, \FFD and \WFDIP outperform state-of-the-art by
2--3 orders of magnitude.

% \subsection{Future Work}
% \label{section:future_work}
In the future, we plan to integrate the general maintenance mechanism with
\emph{EKF}-based SLAM implementations, which we hope will lead to further
improvements. We also plan to begin investigation of novel exploration policies,
based on real-time frontier-detection. 


% \subsection{Speeding-Up \WFD Even Further}\label{section:wfd_speedup}
% \WFD's execution time can be boosted even more by
% reducing the grid size (i.e., by using a coarse grid). Of course, there is a
% trade-off between shorter execution time and the quality of the output frontiers. Even though, standard
% exploration tasks can utilize the output frontiers received in this manner.
% The grid is divided into blocks in size of the robot's width and height.
% Smaller blocks will not make sure that robot will be able to pass through
% terrain obstacles (i.e. corridors). Each block in the real world is represented
% by a single cell in the reduced grid. In order to determine the occupancy value
% of the cell, we examined different strategies. We considered both the speed of
% creating the new grid and the quality of the output.
% %such as: majority, average or
% %sampling the certain cells in the block. 
% We found out that sampling the center of the block edges and the
% block center yields the best results. 
% % \include{algorithms/iterative_frontier_detector_outline}
% We plan to investigate efficient techniques to reduce the grid size while
% considering the quality of the output coarse grid data. 
